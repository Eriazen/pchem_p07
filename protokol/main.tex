\documentclass[12pt]{article}
\usepackage[T1]{fontenc} 
\usepackage{multirow} 
\usepackage{amsmath, graphicx, siunitx, pgfplots, physics}
\usepackage{booktabs}
\usepackage[czech]{babel} 
\usepackage[utf8]{inputenc}
\usepackage{hyperref}
\usepackage{indentfirst}
\usepackage{float} 
\usepackage{appendix}
\usepackage{listings}
\usepackage{xcolor}
\usepackage[a4paper,left=2cm,right=2cm,top=2.5cm,bottom=2.5cm]{geometry}

\setlength{\heavyrulewidth}{1.8pt}
\setlength{\lightrulewidth}{1.2pt} 
\setlength{\parindent}{1cm}
\setlength{\parindent}{24pt}
\setlength{\parskip}{0em}

% bibliography spacing
\def\bstep{-0.2cm}

% listing settings
\definecolor{codegreen}{rgb}{0,0.6,0}
\definecolor{codegray}{rgb}{0.5,0.5,0.5}
\definecolor{codepurple}{rgb}{0.58,0,0.82}
\definecolor{backcolour}{rgb}{0.95,0.95,0.92}

\lstdefinestyle{python}{
    %backgroundcolor=\color{backcolour},   
    commentstyle=\color{codegreen},
    keywordstyle=\color{magenta},
    numberstyle=\small\color{codegray},
    stringstyle=\color{codepurple},
    basicstyle=\ttfamily\small,
    morekeywords=
    breakatwhitespace=false,         
    breaklines=true,
    captionpos=b,
    keepspaces=true,                 
    numbers=left,
    numbersep=5pt,                  
    showspaces=false,                
    showstringspaces=false,
    showtabs=false,                  
    tabsize=2,
    frame=tb,
    inputencoding=utf8,
    extendedchars=true
}
\lstset{style=python}

\begin{document}
	\begin{titlepage}
		
		\vspace*{1cm}
		\begin{center} 
			\begin{tabular}{ p{3.75cm} | p{8cm} p{2cm} }
				\multicolumn{3}{c}{\Large \textbf{Protokol}} \\
				
				\toprule
				
				Instituce & Vysoká škola chemicko-technologická, Praha & \multirow{3}{=}[-0.1\baselineskip]{\centering \includegraphics[width=1.2cm, height=1.2cm]{logoVSCHT_ikona.png}} \\
				
				Fakulta & Fakulta chemicko-inženýrská & \\
				
				Katedra & Ústav fyzikální chemie (403) & \\
				
				Předmět & \multicolumn{2}{l}{Počítačová chemie} \\
				
				Kód předmětu & \multicolumn{2}{l}{B403011} \\
				
				\midrule 
				
				Kód projektu & \multicolumn{2}{l}{P07-DOPRAVA} \\
				
				Název projektu & \multicolumn{2}{l}{Dopravní zácpa} \\
				
				Datum vypracování & \multicolumn{2}{l}{30. 12. 2025} \\
				
				\midrule
				
				Jméno a přijmení & \multicolumn{2}{l}{Vít Večerník} \\
				
				Kruh & \multicolumn{2}{l}{359} \\
				
				\bottomrule 
			\end{tabular}
		\end{center}
		
	\end{titlepage}
	
	\section{Úvod}
    Dopravní inženýrství a teorie dopravy se zabývají matematickým popisem chování různých vozidel na silnici. Jedním z nejvíce studovaných problémů v této oblasti je fantomová dopravní zácpa. Jedná se o situaci, kdy dojde ke zpomalení či zastavení provozu bez jednoznačné vnější příčiny, jako je např. dopravní nehoda či práce na silnici.

    Fantomové zácpy vznikají v důsledku nelineárních interakcí mezi jednotlivými řidiči.
    Jedná se o problém mnoha těles, kde jsou malé změny v rychlosti jednoho vozidla zesíleny reakcí ostatních vozidel.
    Pokud je tok dopravy nestabilní, pak i malé přibrzdění vedoucího vozidla donutí řidiče v závěsu brzdit o něco intenzivněji, z důvodu reakčního času a snaze udržet bezpečný odstup.
    Tento jev se šíří kolonou směrem dozadu a postupně narůstá až k úplnému zastavení vozidel~\cite{treiber2013}.
    Výsledkem je tzv. “stop-and-go” vlna, která se prostorem šíří proti směru jízdy vozidel.

    Jev fantomové dopravní zácpy byl experimentálně prokázán v kontrolovaných podmínkách~\cite{sugiyama2008}.
    V klíčovém experimentu bylo na dráhu dlouhou 230 m umístěno 22 vozidel.
    Každý řidič měl udržovat konstantní rychlost 30 km/h. Přestože na dráze nebyli žádné překážky.
    Po krátké době došlo ke spontánnímu narušení plynulosti a vzniku shluku stojících vozidel, který putoval proti směru jízdy.
    Experiment tedy potvrdil, že zácpy jsou inherentní vlastností dopravního toku při překročení určité kritické hustoty vozidel.

    Pro simulaci tohoto chování se využívají tzv. mikroskopické modely dopravy, které popisují pohyb každého vozidla zvlášť.
    To je v rozporu s makroskopickými modely, které simulují dopravu jako kontinuum vozidel a pro tento druh simulací se nehodí.
    Nejčastěji využívanou skupinou jsou modely sledování vozidla (car-following model, CFM).
    Ty jsou založeny na soustavě obyčejných diferenciálních rovnic, kde zrychlení vozidla je funkcí jeho rychlosti, vzdálenosti od vozidla před ním a rozdílu jejich rychlostí.
    Každé vozidlo si chce udržet co nejkratší, ale zároveň bezpečnou vzdálenost od předešlého vozidla.

    Jedním ze známých CFM je model optimální rychlosti (optimal velocity model, OMV) představený v~\cite{bando1995}.
    Tento model zavádí funkci optimální rychlosti, která závisí výhradně na vzdálenosti mezi vozidly.
    Přestože se jedná o relativně jednoduchý model, dokáže spolehlivě reprodukovat nestabilitu toku dopravy.

    Další známý CFM je model chytrého řidiče (intelligent driver model, IDM) představený v~\cite{treiber2000}.
    IDM je pokročilejší model, který bere v úvahu nejen snahu dosáhnou požadované rychlosti, ale také strategii brždění pro udržení bezpečného odstupu.
    Produkuje realistické profily zrychlení a je standardem v dopravních simulacích.

    Alternativou ke spojitým CFM jsou diskrétní modely celulárních automatů, např. model Nagel-Schreckenberg~\cite{nagel1992}.
    Tyto modely rozdělují silnici a čas na diskrétní kroky a zavádí prvek náhody, který vytváří prvotní poruchy v homogenním toku.

	
	\section{Cíle}
	Hlavním cílem projektu je vytvořit počítačovou simulaci silničního provozu, která demonstruje vznik fantomové dopravní zácpy vlivem malých náhodných změn jízdy v koloně.
    Za tímto účelem byly stanoveny dílčí cíle práce:
    \newpage
    \begin{enumerate}
        \item Vytvořit fyzikální model pohybu vozidla, který zahrnuje rovnoměrné zrychlení ke stanovené maximální rychlosti a logiku zpomalování v závislosti na vzdálenosti vozidla vpředu.
        \item Zahrnutí lidského faktoru a náhody v podobě reakčního času, respektive náhodných změn rychlosti vozidla o malé hodnoty.
        \item Implementace simulace v periodických okrajových podmínkách odpovídajích experimentu viz~\cite{sugiyama2008}.
        \item Analýza vzniku kolon za různých počátečních podmínek a parametrů. Potvrzení, že při vhodném nastavení dojde ke spontannímu vzniku fantomové kolony.
    \end{enumerate}

	\section{Postup}
    
	\begin{lstlisting}
        # Python code test
        if x < 10:
            print('x je mensi nez 10')
    \end{lstlisting}
	\section{Výsledky}
	
	\section{Závěr}

    \begin{thebibliography}{9}
        { \normalsize
        
        \bibitem{bando1995} BANDO, M.; K. HASEBE; A. NAKAYAMA; A. SHIBATA a Y. SUGIYAMA, 1995. Dynamical model of traffic congestion and numerical simulation. \textit{Physical Review E}. Online. 51(2), 1035–1042. Dostupné z: \url{https://doi.org/10.1103/PhysRevE.51.1035}
        \bibitem{nagel1992} NAGEL, Kai a Michael SCHRECKENBERG, 1992. A cellular automaton model for freeway traffic. \textit{Journal de Physique I}. Online. 2, 2221. Dostupné z: \url{https://doi.org/10.1051/jp1:1992277}
        \bibitem{sugiyama2008} SUGIYAMA, Yuki; Minoru FUKUI; Macoto KIKUCHI; Katsuya HASEBE; Akihiro NAKAYAMA; Katsuhiro NISHINARI; Shin-ichi TADAKI a Satoshi YUKAWA, 2008. Traffic jams without bottlenecks - experimental evidence for the physical mechanism of the formation of a jam. \textit{New Journal of Physics}. Online. 10, 33001. Dostupné z: \url{https://doi.org/10.1088/1367-2630/10/3/033001}
        \bibitem{treiber2013} TREIBER, Martin, 2013. \textit{Traffic Flow Dynamics}. Online. ISBN 978-3-642-32459-8. Dostupné z: \url{https://doi.org/10.1007/978-3-642-32460-4}
        \bibitem{treiber2000} TREIBER, Martin; Ansgar HENNECKE a Dirk HELBING, 2000. Congested traffic states in empirical observations and microscopic simulations. \textit{Physical Review E}. Online. 62(2), 1805–1824. Dostupné z: \url{https://doi.org/10.1103/PhysRevE.62.1805}
        }
    \end{thebibliography}
	
\end{document}